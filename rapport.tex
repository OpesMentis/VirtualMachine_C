\documentclass[11pt]{article}

\usepackage[T1]{fontenc}
\usepackage[utf8]{inputenc}
\usepackage[francais]{babel}
\usepackage[scale=1, top=2cm, bottom=2cm, left=2cm, right=2cm]{geometry}

\begin{document}
	\title{Architecture numérique}
	\author{Antoine \bsc{Planchot} et Julien \bsc{Tomezach}}
	\maketitle

	\part*{Introduction}
	\paragraph{}
	L'objectif de ce projet est la réalisation d'une machine virtuelle d'un microprocesseur en langage C. Le tout se décompose en deux programmes comme autant d'étapes nécessaire à l'accomplissement de cet objectif. Le tout se décompose donc comme suit~:

	\begin{itemize}
		\item la conversion d'un code en assembleur dans un fichier texte quelconque vers le fichier binaire~;
		\item la conversion du fichier binaire vers l'exécution de la machine virtuelle, en C.
	\end{itemize}

	\paragraph{}
	Une nomenclature commune ayant été établie, à laquelle on pourra se référer à tout moment dans la partie dédiée, nous pûmes nous mettre au travail.

	\part*{Encodage des informations binaires}
	Non scholae sed vitae discimus.

	\part*{De l'assembleur vers le binaire}
	Ave Maria gratia plena.

	\part*{Du binaire vers la machine virtuelle}
	Beati pauperes spiritu.

	\part*{Conclusion}
	Cave canem.

\end{document}
