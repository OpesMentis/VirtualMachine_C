\documentclass[11pt]{article}

\usepackage[T1]{fontenc}
\usepackage[utf8]{inputenc}
\usepackage[francais]{babel}
\usepackage[scale=1, top=2cm, bottom=2cm, left=2cm, right=2cm]{geometry}

\begin{document}
	\title{Architecture numérique}
	\author{Antoine \bsc{Planchot} et Julien \bsc{Tomezach}}
	\maketitle

	\part*{Introduction}
	\paragraph{}
	L'objectif de ce projet est la réalisation d'une machine virtuelle d'un microprocesseur en langage C. Le tout se décompose en deux programmes comme autant d'étapes nécessaire à l'accomplissement de cet objectif. Le tout se décompose donc comme suit~:

	\begin{itemize}
		\item la conversion d'un code en assembleur dans un fichier texte quelconque vers le fichier binaire~;
		\item la conversion du fichier binaire vers l'exécution de la machine virtuelle, en C.
	\end{itemize}

	\paragraph{}
	Une nomenclature commune ayant été établie, à laquelle on pourra se référer à tout moment dans la partie dédiée, nous pûmes nous mettre au travail.

	\part*{Encodage des informations binaires}
	Non scholae sed vitae discimus.

	\part*{De l'assembleur vers le binaire}
	Outre les problèmes inhérents au langage C, nous n'avons n'a pas rencontré de problème majeur, si ce n'est qu'il a fallu du temps à l'auteur avant de créer un code propre s'adaptant à toutes les situations possibles. 
En effet, lorsque l'on lit le code assembleur , suivant la commande, il n'y aura pas le même nombre d'argument et ces arguments ne seront pas rencontrés au même endroit. Dés lors une approche générique semble difficile dans la traduction en binaire des lignes de commandes. Néanmoins on arrive à la fin à traduire chaque commande on agençant différemment des fonctions de haut niveau (add_register, add_flag_argO) rajoutant les bits de registre, de flag et du paramètre o. On reconnaîtra ainsi 5 grands type de commande se traduisant de la même façon : add like, jump, braz like, scall like et stop.
Lorsque les bits seront non lu, on les mets à un. Sauf lorsque l'on complète si besoin la fin de l'instruction par des 0.

	\part*{Du binaire vers la machine virtuelle}
	Beati pauperes spiritu.

	\part*{Conclusion}
	Cave canem.

\end{document}
